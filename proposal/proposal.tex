\documentclass[11pt,a4paper]{article}

\usepackage{float}
\usepackage{hyperref}

\usepackage{listings}
\usepackage[usenames,dvipsnames]{color}
\lstset{
	language=Java,
	basicstyle=\small\ttfamily,
	keywordstyle=\color{Blue}\bf,
	tabsize=2,
	frame=single
}

\title{HIVE-13680: Provide a way to compress ResultSets}
\author{Kevin Liew}

\begin{document}

\maketitle

\begin{abstract}
	Hive data pipelines invariably involve JDBC/ODBC drivers which access Hive through its Thrift interface. 
	Consequently, any enhancement to the Thrift vector will benefit all users.
	
	Prior to 
	\href{https://issues.apache.org/jira/browse/HIVE-12049}{HIVE-12049}
	, HiveServer2 would read a full ResultSet from HDFS before deserializing and re-serializing into Thrift objects for RPC transfer.
	Following our enhancement, task nodes serialize Thrift objects from their own block of a ResultSet file. 
	HiveServer2 reads the Thrift output and transfers it to the remote client. 
	This parallel serialization strategy reduced latency in the data pipeline.
	
	However, network capacity is often the most scarce resource in a system. 
	As a further enhancement, network load can be eased by having task nodes compress their own block of a ResultSet as part of the serialization process.
\end{abstract}

\section{Introduction}
	The changes proposed herein draw from Rohit Dholakia's design document and patches for
	\href{https://issues.apache.org/jira/browse/HIVE-10438}{HIVE-10438}, which implemented compression on HiveServer2.
	Now that
	\href{https://issues.apache.org/jira/browse/HIVE-12049}{HIVE-12049}
	has been committed, compression can take place in parallel on the task nodes.
	
	Our goals for this enhancement are to:
	\begin{itemize}
		\item improve performance out-of-the-box for new clients
		\item maintain compatibility with old clients
		\item provide flexibility yet security
		\item confer a simple interface
	\end{itemize}
	
\section{Design Overview}

	\subsection{Compressor-Decompressor Interface}
		We define a compressor interface which must be implemented by compressor plugins.
		A default compressor will compress all data-types using the Snappy algorithm.
		Type-specific compression can be achieved by implementing a compressor that delegates compression to other compressors based on a case-switch block.
		Compressors may also pass specific data-types through uncompressed.
		
		\begin{lstlisting}[title=org.apache.hive.service.cli.CompDe; CompDe.java,gobble=6,otherkeywords={ColumnBuffer,Object}]
			@InterfaceAudience.Private
			@InterfaceStability.Stable
			public interface ColumnCompDe {
				public byte[] compress(ColumnBuffer columns);
				public Object decompress(byte[] columnsBlob);
			}
		\end{lstlisting}
		
		Operating in the final task nodes, `compress` takes a batch of rows contained in a ColumnBuffer (number of rows will be equal to or less than \linebreak hive.server2.thrift.resultset.max.fetch.size) and outputs a binary blob.
		The compressor is free to pack additional details such as look-up tables within this blob.
		
		The client receives results batch-by-batch, calling `decompress` on each compressed column to receive an Object containing the rows in that column-batch which, along with `nulls` and the type-information in TEnColumn, will be used to construct a ColumnBuffer.
		
	\subsection{Client-Server Negotiation}
		Upon connection, the client will send a list of available compressors. The server will reply with a list of compressors ordered by preference.
		Both the client and server will choose the first compressor in the server's list that also exists in the client's list.
		All results for that session will be compressed using that compressor.
		The results will not be compressed if the client and server cannot agree on a compressor, or if either the client or server has configured an empty compressor list.
		This negotiation scheme will maintain compatibility with old clients while using compression whenever it is available.
		
	\subsection{Configuration options}
		\begin{table}[H]
			\begin{tabular}{| p{3cm} | c | p{6.5cm} |} \hline
				\textbf{Option} & \textbf{Default} & \textbf{Description} \\ \hline
				hive.resultSet\linebreak
				.compressor\linebreak
				.list
				& snappy & A comma-separated list of compressors that the server can use, ordered by preference.
				\\ \hline
				hive.server2\linebreak
				.thrift.resultset\linebreak
				.max.fetch.size
				& 1000 & Max number of rows sent in one Fetch RPC call by the server to the client.
				\\ \hline
			\end{tabular}
			\caption{Server configuration options}
		\end{table}

		\begin{table}[H]
			\begin{tabular}{| p{3cm} | c | p{6.5cm} |} \hline
				\textbf{Parameter} & \textbf{Default} & \textbf{Description} \\ \hline
				compressors
				& snappy & A comma-separated list of compressors that the client can use. 
				\\ \hline
			\end{tabular}
			\caption{Client connection parameters}
		\end{table}
		
\section{Implementation}
	
	\subsection{Compression}
		The output of a final node in a DAG is either a set of rows (from a map task) or a single value (from a reduce task).
		Following
		\href{https://issues.apache.org/jira/browse/HIVE-12049}{HIVE-12049}
		, the output is buffered row-by-row into TColumns in TRowSet and serialized in batches to a file in HDFS.
		The final output file is read by HiveServer2 (deserializing the columns as a binary blob) and sent to the client.
		
		Compressors will operate on the batch-level in ThriftJDBCBinarySerDe.
		In the event that a column is not compressible, the column will be serialized as an uncompressed column.
		Compressed and uncompressed columns will be serialized contiguously in the output file. A bit-mask will be created to indicate which columns were compressed successfully.
		This allows for a robust system where each column will have a compression state per-batch. If any column fails to compress in one batch, that column can still be compressed in other batches.
		
		
	\subsection{Decompression}
		Results are serialized in batches.
		Consequently, the client must deserialize batch-by-batch in ColumnBasedSet.
		For each batch: the client starts with a check binary with a bit value for `1` and does a bitwise `AND` with the compressed-column bitmask.
		If the result is equal to the starting bit, then the column is compressed.
		The client reads either a compressed or uncompressed column into the appropriate Thrift object which is converted into a ColumnBuffer.
		This is done for each column, with the check bit shifted to the left by one to check the next column for that iteration.
		The end result is a ColumnBasedSet for each batch.
	
	\subsection{Thrift Structures}
		
		\begin{lstlisting}[title=TCLIService.thrift,gobble=8,otherkeywords={binary,i32,i64,string,struct,TColumn,TEnColumn,TRow,TRowSet,TTypeId}]
			//Represents an encoded column
			struct TEnColumn {
				1: required binary enData
				2: required binary nulls
				3: required TTypeId type
			}
			
			// Represents a rowset
			struct TRowSet {
				// The starting row offset of this rowset.
				1: required i64 startRowOffset
				2: required list<TRow> rows
				3: optional list<TColumn> columns
				4: optional binary binaryColumns
				5: optional i32 columnCount
				6: optional list<TEnColumn> enColumns
				7: optional string compressorName
				8: optional binary compressorMask
			}
		\end{lstlisting}
		
		TRowSet will have a list of TEnColumn to store compressed columns. The compressor-name and a bitmask to track compressed columns will also be added.
		Otherwise, the Thrift structures used to support compression are largely unchanged from 
		\href{https://issues.apache.org/jira/browse/HIVE-10438}{HIVE-10438}
		. They are described here for completeness.
		
		TEnColumns `enData` is a binary blob containing one compressed row-batch of a column.
		`nulls` is a bitmap indicating null rows in the column.
		`type` is the column's data-type.
		
		TRowSet `startRowOffet` and `rows` are deprecated following \href{https://issues.apache.org/jira/browse/HIVE-3746}{HIVE-3746}.
		Result files are now column-oriented and either in `binaryColumns` (in the task nodes and on HiveServer2) or a combination of `columns` and `enColumns` (in the client).
		`compressorName` indicates the compressor plugin that was used to compress the result set.
		`compressorMask` is a bit-mask indicating compressed columns.
		
\end{document}
