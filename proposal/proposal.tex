\documentclass[11pt,a4paper]{article}

\usepackage{enumerate}
\usepackage{float}
\usepackage{hyperref}

\usepackage{listings}
\usepackage[usenames,dvipsnames]{color}
\lstset{
	language=Java,
	basicstyle=\small\ttfamily,
	keywordstyle=\color{Blue}\bf,
	tabsize=2,
	frame=single
}

\title{HIVE-13680: Provide a way to compress ResultSets}
\author{Kevin Liew}

\begin{document}

\maketitle

\begin{abstract}
	
	Hive data pipelines invariably involve JDBC/ODBC drivers which access Hive through its Thrift interface. 
	Consequently, any enhancement to the Thrift vector will benefit all users.
	
	Prior to 
	\href{https://issues.apache.org/jira/browse/HIVE-12049}{HIVE-12049},
	HiveServer2 would read a full ResultSet from HDFS before deserializing and re-serializing into Thrift objects for RPC transfer.
	Following our enhancement, task nodes serialize Thrift objects from their own block of a ResultSet file. 
	HiveServer2 reads the Thrift output and transfers it to the remote client. 
	This parallel serialization strategy reduced latency in the data pipeline.
	
	However, network capacity is often the most scarce resource in a system. 
	As a further enhancement, network load can be eased by having task nodes compress their own block of a ResultSet as part of the serialization process.
	
\end{abstract}

\section{Introduction}
	
	The changes proposed herein draw from Rohit Dholakia's design document and patches for
	\href{https://issues.apache.org/jira/browse/HIVE-10438}{HIVE-10438}, which implemented compression on HiveServer2.
	Now that
	\href{https://issues.apache.org/jira/browse/HIVE-12049}{HIVE-12049}
	has been committed, compression can take place in parallel on the task nodes.
	
	Our goals for this enhancement are to:
	
	\begin{itemize}
		\item improve performance out-of-the-box for new clients
		\item maintain compatibility with old clients
		\item provide flexibility yet security
		\item confer a simple interface	
	\end{itemize}
	
\section{Design Overview}

	\subsection{Compressor-Decompressor Interface}
		
		We define a compressor-decompressor (CompDe) interface which must be implemented by CompDe plugins.
		A default CompDe will process all data-types using the Snappy algorithm.
		Type-specific compression-decompression can be implemented by the CompDe.
		CompDes may also pass specific data-types through unprocessed.
		
		\begin{lstlisting}[
		title=org.apache.hive.service.cli.CompDe;CompDe.java,
		gobble=6,
		otherkeywords={ColumnBasedSet,HashSet}]
			@InterfaceAudience.Private
			@InterfaceStability.Stable
			public interface ColumnCompDe {
				public boolean init(HashSet config);
				public byte[] compress(ColumnBuffer[] colSet);
				public ColumnBuffer[] decompress(byte[] compressedSet);
			}
		\end{lstlisting}
		
		Operating in the final task nodes, `compress' takes a batch of rows contained in a ColumnBuffer array (number of rows will be equal to \linebreak hive.server2.thrift.resultset.max.fetch.size except for the last batch) and outputs a binary blob.
		The compressor is free to pack additional details such as look-up tables within this blob.
		
		The client receives results batch-by-batch, calling `decompress' on each blob to receive an array of ColumnBuffer.
		
	\subsection{Configuration options}
		
		\begin{table}[H]
			\begin{tabular}{| p{2.7cm} | p{1.9cm} | p{6.4cm} |} \hline
				
				\textbf{Option} & \textbf{Default} & \textbf{Description} \\ \hline
				
				hive.server2\linebreak
				.thrift.resultset\linebreak
				.serialize.in.tasks
				& false
				& Whether we should serialize the Thrift structures used in JDBC ResultSet RPC in task nodes.\linebreak
				We use SequenceFile and ThriftJDBCBinarySerDe to read and write the final results if this is true.\linebreak
				(Required for compression)
				\\ \hline
				
				hive.server2\linebreak
				.thrift.resultset\linebreak
				.max.fetch.size
				& 1000
				& Max number of rows sent in one Fetch RPC call by the server to the client.\linebreak
				(Determines rows in a compressed batch)
				\\ \hline
				
				hive.resultSet\linebreak
				.compressors
				& snappy
				& A list of keywords for available compressors, ordered by preference.
				\\ \hline
				
				hive.resultset\linebreak
				.compressor\linebreak
				.snappy.class
				& org.apache\linebreak.hive\linebreak.resultset\linebreak.compressor\linebreak.snappy
				& Map the snappy compressor keyword to a Java class.
				\\ \hline
				
				hive.resultset\linebreak
				.compressor\linebreak
				.\textless name\textgreater .class
				&
				& Map the \textless name\textgreater\space compressor keyword to a Java class.
				\\ \hline
				
				hive.resultset\linebreak
				.compressor\linebreak
				.\textless name\textgreater\linebreak.\textless param\textgreater
				&
				& A default option for \textless param\textgreater\space for the \textless name\textgreater\space compressor. 
				\\ \hline
				
			\end{tabular}
			\caption{Server CompDe configuration options}
		\end{table}
		
		\begin{table}[H]
			\begin{tabular}{| p{3cm} | c | p{6.5cm} |} \hline
				
				\textbf{Parameter} & \textbf{Default} & \textbf{Description} \\ \hline
				compressors
				& snappy
				& A hiveConfs list of compressor keywords that the client can use. 
				\\ \hline
				
				compressor\linebreak
				.\textless name\textgreater\linebreak
				.\textless variable\textgreater
				&
				& A hiveConfs to override the default compressor config. 
				\\ \hline
				
			\end{tabular}
			\caption{Client connection parameters}
		\end{table}
		
	\subsection{Client-Server Negotiation}
		
		CompDe negotiation proceeds as follows:
		
		\begin{enumerate}
			
			\item the client optionally notifies the server of available compressors by using `hiveConfs' parameters in the connection string
			\begin{itemize}
				\item a list of compressor keywords
				\item parameters to override the server-configured compressor defaults for each plugin
			\end{itemize}
			
			\item the server creates a list of compressors from the intersection of the client and server configured plugins, ordered by the server's preference
			
			\item for each plugin in the intersection
			\begin{enumerate}[1.]
				\item the server merges the server-configured defaults with the client overrides into a HashSet
				\item the server passes the merged config to the `init' function of the compressor
			\end{enumerate}
			
			\item the server notifies the client of the keyword for the first CompDe that successfully initialized
		\end{enumerate}
		
		All result sets will be compressed using that CompDe.
		The results will not be compressed if the client and server cannot agree on a compressor, if all of the available compressors fail to initialize with merged config, or if either the client or server has configured an empty compressor list.
		This negotiation scheme will maintain compatibility with old clients while using compression whenever it is available.
		
\section{Implementation}
	
	\subsection{Thrift Structures}
		
		\begin{lstlisting}[
		title=TCLIService.thrift,
		gobble=6,
		otherkeywords={binary,i32,i64,string,struct,TColumn,TCompressorConfig,TRow,TRowSet}]
			// Represents a rowset
			struct TRowSet {
				// The starting row offset of this rowset.
				1: required i64 startRowOffset
				2: required list<TRow> rows
				3: optional list<TColumn> columns
				4: optional binary binaryColumns
				5: optional i32 columnCount
			}
		\end{lstlisting}
		
		TRowSet is structurally unchanged by this enhancement. It is described here for completeness.
		
		`startRowOffet' and `rows` are deprecated in favor of `columns' following \href{https://issues.apache.org/jira/browse/HIVE-3746}{HIVE-3746}.
		\href{https://issues.apache.org/jira/browse/HIVE-10438}{HIVE-10438}
		added an option `hive.server2.thrift.resultset-.serialize.in.tasks' to serialize columns (in batches, and in parallel on task nodes) to `binaryColumns' as contiguous TColumn. Following this enhancement, 
		\href{https://issues.apache.org/jira/browse/HIVE-13680}{HIVE-13680},
		`binaryColumns' will contain contiguous batches of the CompDes `compress' function output (compressed sets of columns) when compression is enabled.
		
	\subsection{Compression}
	
		The output of a final node in a DAG is either a set of rows (from a map task) or a single value (from a reduce task).
		Following
		\href{https://issues.apache.org/jira/browse/HIVE-12049}{HIVE-12049}
		, the output is buffered row-by-row into TColumns in TRowSet and serialized in batches to a file in HDFS.
		The final output file is read by HiveServer2 and sent to the client.
		
		Compressors will operate on the batch-level (`serializeBatch') in ThriftJDBCBinarySerDe to output binary which is serialized to `binaryColumns' of TRowSet via `TCompactProtocol.writeBinary'.
		We have elected to give the CompDe access to the full column set rather than compressing column-by-column as this allows for optimizations such as grouping similar columns under the same compression algorithm.
		
		Currently, if the user has enabled `hive.exec.compress.output', the setting `mapred.output.compression.codec' (regardless of the execution engine) is checked by Hive's implementation of a Hadoop RecordWriter to determine the codec used to compress the final SequenceFile.
		However, this enhancement will have ThriftJDBCBinarySerDe compress data inside `binaryColumns' which is serialized to a SequenceFile.
		Applying additional compression on `binaryColumns' in the SequenceFile will have diminished marginal returns while wasting cpu cycles.
		Consequently, when a compressor plugin is configured,  `hive.exec.compress.output' will be disabled before serializing the final output.
		However, `hive.exec.compress.intermediate' and associated options are interoperable with this enhancement.
		
	\subsection{Decompression}
	
		Results are serialized in batches.
		Consequently, the client must deserialize batch-by-batch in ColumnBasedSet.
		`decompress' is called on `binaryColumns' to get an array of ColumnBuffer.
		
\end{document}
