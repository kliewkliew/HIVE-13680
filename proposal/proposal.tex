\documentclass[11pt,a4paper]{article}
\usepackage{hyperref}

\title{HIVE-13680: Provide a way to compress ResultSets}
\author{Kevin Liew}

\begin{document}

\maketitle

\begin{abstract}
	Hive data pipelines invariably involve JDBC/ODBC drivers which access Hive through its Thrift interface. 
	Consequently, any enhancement to the Hive's Thrift data pipeline will benefit all users.
	
	Prior to \href{https://issues.apache.org/jira/browse/HIVE-12049}{HIVE-12049}
	, HiveServer2 would read a full ResultSet from HDFS before deserializing and re-serializing into Thrift objects for RPC transfer.
	Following our enhancement, task nodes serialize Thrift objects from their own block of a ResultSet file. 
	HiveServer2 simply transfers these objects to a remote client. 
	This parallel deserialization strategy reduced latency in the data pipeline.
	
	However, network load is often the most scarce resource in a system. 
	As a further enhancement, this load can be eased by having task nodes compress their own block of a ResultSet as part of the serialization process.
\end{abstract}

\section{Introduction}
	The changes proposed herein draw from the design document of 
	\href{https://issues.apache.org/jira/browse/HIVE-12049}{HIVE-12049}, proposed by Rohit Dholakia.
	
	This enhancement should improve performance without any configuration while providing flexibility yet through a simple interface.

\section{Proposed Changes}

	\subsection{Compressed RowSet Structure}
		\subsubsection{Thrift Object}
		\subsubsection{HiveServer2 Class}
		TODO

	\subsection{Interface}
		\subsubsection{Compressor Interface}
			We will define a compressor interface which must be implemented by compressor plugins.
			A Snappy compressor will compress all data-types using the Snappy algorithm.
			Type-specific compression can be achieved by implementing a compressor that delegates compression to other compressors based on a case-switch block.
		
		
		\subsubsection{Client-Server Negotiation}
			The client will specify a preferred compressor upon connection. 
			If the server does not have that compressor plugin, it will request the jar from the client. 
			The client should send the compressor plugin and all dependencies.
		
			To prevent breaking compatibility with old clients, the server will not compress results unless the client has requested compression.
			However, new versions of beeline will try to negotiate a compressor scheme by default.
		
		\subsubsection{Server Security}
			We place the onus of security on the server administrator, recommending Kerberos authentication with user-impersonation to limit the jar file to the same permissions as the authenticated user. 
			The server administrator also has the option to disable compressor plugins altogether.
		
		
	\subsection{Configuration options}
	\begin{table}[!h]
		\begin{tabular}{| l | l | p{4cm} |} \hline
			\textbf{Option} & \textbf{Default} & \textbf{Description} \\ \hline
			hive.resultSet.compressor.enabled & true & Enable or disable compression \\ \hline
			%hive.resultSet.compressor & Snappy & The default compressor when the client has not specified one \\ \hline
		\end{tabular}
		\caption{Configuration options}
	\end{table}

\section{Type-Specific Compressors}
	TODO

\end{document}
