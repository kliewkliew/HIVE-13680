\documentclass[11pt,a4paper]{article}
\usepackage{hyperref}

\title{Design Considerations}
\author{Kevin Liew}

\begin{document}

\maketitle

\section{RPC}
	\paragraph{How are ResultSets transferred? Batches contained in Thrift objects?}
	Contiguous batches of TRowSet.
	\begin{verbatim}
		ie.
		ResultSet{
		    TRowSet batch1{
		        TColumn1{row1, row2},
		        TColumn2{row1, row2},
		        columnCount = 2
		    }
		    TRowSet batch2{
		        TColumn1{row3, row4},
		        TColumn2{row3, row4},
		        columnCount = 2
		    }
		}
		
	\end{verbatim}
	except that the columns are in `binaryColumns` rather than a list of TColumn after \href{https://issues.apache.org/jira/browse/HIVE-12049}{HIVE-12049}

\section{Interface}
	\paragraph{Should the preferred compressor be per-connection, or per-query?}

	\paragraph{Should the client send a preferred compressor plugin's jar file? Or should the client specify the compressor in the connection string and require that the server already have the plugin?}
		
	\paragraph{Location of exemplar code for sending jars from the client and receiving at the server?}
	
	\paragraph{How can we give the server more control over plugins?}
	A list of disallowed compressors is not useful if we allow the client to send arbitrary jars with arbitrary entry-classes. If we instead have a list of allowed compressors, this list would require the server to have prior knowledge of all possible compressors. In this case, the server might as well be pre-loaded with those plugins and there is no need for the client to send the jars.

\section{Compressor-Decompressor}
	\paragraph{How do we handle the case where the compressor throws an exception while handling an inner batch?)}
	Do we restart serialization and serialize the batch as an uncompressed column?
	What do we do with prior batches that were already serialized?
	
	\paragraph{Where does decompression occur in the client? Will we operate on (Encoded)ColumnBasedSet or directly on TRowSet?}

\end{document}
